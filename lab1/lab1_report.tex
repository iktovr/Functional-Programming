\documentclass[12pt]{article}

\usepackage{fullpage}
\usepackage{multicol,multirow}
\usepackage{tabularx}
\usepackage{ulem}
\usepackage[utf8]{inputenc}
\usepackage[russian]{babel}
\usepackage{amsmath}
\usepackage{amssymb}

\usepackage{titlesec}

\titleformat{\section}
  {\normalfont\Large\bfseries}{\thesection.}{0.3em}{}

\titleformat{\subsection}
  {\normalfont\large\bfseries}{\thesubsection.}{0.3em}{}

\titlespacing{\section}{0pt}{*2}{*2}
\titlespacing{\subsection}{0pt}{*1}{*1}
\titlespacing{\subsubsection}{0pt}{*0}{*0}
\usepackage{listings}
\lstloadlanguages{Lisp}
\lstset{extendedchars=false,
	breaklines=true,
	breakatwhitespace=true,
	keepspaces = true,
	tabsize=2
}

\begin{document}

\section*{Отчет по лабораторной работе №\,1 \\
по курсу \guillemotleft  Функциональное программирование\guillemotright}
\begin{flushright}
Студент группы 8О-307 МАИ \textit{Бирюков Виктор}, \textnumero 2 по списку \\
\makebox[7cm]{Контакты: {\tt vikvladbir@mail.ru} \hfill} \\
\makebox[7cm]{Работа выполнена: 11.03.2022 \hfill} \\
\ \\
Преподаватель: Иванов Дмитрий Анатольевич, доц. каф. 806 \\
\makebox[7cm]{Отчет сдан: \hfill} \\
\makebox[7cm]{Итоговая оценка: \hfill} \\
\makebox[7cm]{Подпись преподавателя: \hfill} \\

\end{flushright}

\section{Тема работы}
Примитивные функции и особые операторы Коммон Лисп.

\section{Цель работы}
Научиться вводить S-выражения в Лисп-систему, определять переменные и функции, работать с условными операторами, работать с числами, используя схему линейной и древовидной рекурсии.

\section{Задание (вариант №1.45)}
С помощью формулы $((x / y^2 + 2*y) / 3)$ запрограммируйте на языке Коммон Лисп функцию для вычисления кубического корня. Причем $y$ является приближением к кубическому корню из $x$.

Использовать функции {\tt good-enough-p}, {\tt improve} и {\tt cube}.

\section{Оборудование студента}
Процессор AMD Ryzen 7 3700U\,@\,2.3GHz, память: 20Gb, разрядность системы: 64.

\section{Программное обеспечение}
ОС Windows 10, компилятор SBCL 2.2.2, текстовый редактор Sublime Text 4.

\section{Идея, метод, алгоритм}
Функция {\tt cuberoot-iter} вычисляет значение кубического корня методом Ньютона. Приближённое значение проверяется на близость к реальному при помощи предиката {\tt good-enough-p}, если оно недостаточно близко, происходит рекурсивный вызов {\tt cuberoot-iter} со значением, улучшенным функцией {\tt improve}.

В качестве начального приближения используется {\tt 1.0}.

Точность вычислений задается константой {\tt eps}.

\section{Сценарий выполнения работы}

\section{Распечатка программы и её результаты}

\subsection{Исходный код}
\lstinputlisting{./lab1.lisp}

\subsection{Результаты работы}
\begin{lstlisting}
* (cuberoot 27.0)
3.0000007

* (cuberoot 8.0)
2.000005

* (cuberoot 125.0)
5.0

* (cuberoot (cube 127))
127.0
\end{lstlisting}

\section{Дневник отладки}
\begin{tabular}{|c|c|c|c|}
\hline
Дата & Событие & Действие по исправлению & Примечание \\
\hline
\end{tabular}

\section{Замечания автора по существу работы}

Метод Ньютона позволяет довольно быстро найти значение кубического корня с заданной точностью. Однако могут возникнуть проблемы при работе с числами с плавающей точкой. Так, например, на моем устройстве программа не может найти $\sqrt[3]{5}$ c точностью $10^{-7}$

\section{Выводы}
В ходе выполнения лабораторной работы я познакомился с основами языка CommonLisp и реализацией простейших функций.

\end{document}