\documentclass[12pt]{article}

\usepackage{fullpage}
\usepackage{multicol,multirow}
\usepackage{tabularx}
\usepackage{ulem}
\usepackage[utf8]{inputenc}
\usepackage[russian]{babel}
\usepackage{amsmath}
\usepackage{amssymb}

\usepackage{libertine}
\renewcommand{\ttdefault}{cmtt}

\usepackage{titlesec}

\titleformat{\section}
  {\normalfont\Large\bfseries}{\thesection.}{0.3em}{}

\titleformat{\subsection}
  {\normalfont\large\bfseries}{\thesubsection.}{0.3em}{}

\titlespacing{\section}{0pt}{*2}{*2}
\titlespacing{\subsection}{0pt}{*1}{*1}
\titlespacing{\subsubsection}{0pt}{*0}{*0}
\usepackage{listings}
\lstloadlanguages{Lisp}
\lstset{extendedchars=false,
	breaklines=true,
	breakatwhitespace=true,
	keepspaces = true,
	tabsize=2
}

\begin{document}

\section*{Отчет по лабораторной работе №\,5 \\
по курсу \guillemotleft  Функциональное программирование\guillemotright}
\begin{flushright}
Студент группы 8О-307 МАИ \textit{Бирюков Виктор}, \textnumero 2 по списку \\
\makebox[7cm]{Контакты: {\tt vikvladbir@mail.ru} \hfill} \\
\makebox[7cm]{Работа выполнена: 21.05.2022 \hfill} \\
\ \\
Преподаватель: Иванов Дмитрий Анатольевич, доц. каф. 806 \\
\makebox[7cm]{Отчет сдан: \hfill} \\
\makebox[7cm]{Итоговая оценка: \hfill} \\
\makebox[7cm]{Подпись преподавателя: \hfill} \\

\end{flushright}

\section{Тема работы}
Обобщённые функции, методы и классы объектов.

\section{Цель работы}
Научиться определять простейшие классы, порождать экземпляры классов, считывать и изменять значения слотов, научиться определять обобщённые функции и методы.

\section{Задание (вариант №5.43)}
Определите обычную функцию {\tt der-polynom} с одним параметром --- многочленом, т.е. экземпляром класса {\tt polynom}.

Функция должна вычислять производную $P'(x)$.

\section{Оборудование студента}
Процессор AMD Ryzen 7 3700U\,@\,2.3GHz, память: 20Gb, разрядность системы: 64.

\section{Программное обеспечение}
ОС Windows 10, LispWorks Personal Edition 7.1.2.

\section{Идея, метод, алгоритм}
Известно, что производная степени $\frac{d x^n}{d x} = n x^{n-1}$, $n \neq 0$, в то время как $\frac{d x^0}{d x} = \frac{d 1}{d x} = 0$. Таким образом дифференцирование многочлена происходит по следующей схеме:

\begin{itemize}
\setlength{\itemsep}{-1mm}
\item терм вида $a x^n$, $n > 0$ становится термом $a n \cdot x^{n-1}$
\item терм, соответствующий свободному члену, удаляется.
\end{itemize}

Эту операцию над списком термов осуществляет функция {\tt der-terms}. Функция {\tt der-polynom} создает новый многочлен с тем же символом и измененными термами.

\section{Сценарий выполнения работы}

\section{Распечатка программы и её результаты}

\subsection{Исходный код}
\lstinputlisting{./lab5.lisp}

\subsection{Результаты работы}
\begin{lstlisting}[escapechar=\%]
CL-USER 1 > (setq p (make-instance 'polynom
          :var 'x
          :terms (list (make-term :order 2 :coeff 5)
                       (make-term :order 1 :coeff 3.3)
                       (make-term :order 0 :coeff -7))))
[%МЧ% (X) +5X^2+3.3X-7]

CL-USER 2 > (der-polynom p)
[%МЧ% (X) +10X+3.3]

CL-USER 3 > (der-polynom (der-polynom p))
[%МЧ% (X) +10]

CL-USER 4 > (der-polynom (der-polynom (der-polynom p)))
[%МЧ% (X) ]

CL-USER 5 > (setq p (make-instance 'polynom
          :var 'x
          :terms (list (make-term :order 100 :coeff 1))))
[%МЧ% (X) +1X^100]

CL-USER 6 > (der-polynom p)
[%МЧ% (X) +100X^99]

CL-USER 7 > (der-polynom (der-polynom p))
[%МЧ% (X) +9900X^98]

CL-USER 8 > (der-polynom (der-polynom (der-polynom p)))
[%МЧ% (X) +970200X^97]
\end{lstlisting}

\section{Дневник отладки}
\begin{tabular}{|c|c|c|c|}
\hline
Дата & Событие & Действие по исправлению & Примечание \\
\hline
\end{tabular}

\section{Замечания автора по существу работы}
В данной работе я решил использовать {\tt loop} вместо явной рекурсии и это оказалось несколько проще. Сложность алгоритма --- $O(n)$, где $n$ --- количество термов.

\section{Выводы}
В ходе выполнения лабораторной работы я познакомился с созданием простейших классов, а также с реализацией функций для их обработки.

\end{document}