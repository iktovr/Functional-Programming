\documentclass[12pt]{article}

\usepackage{fullpage}
\usepackage{multicol,multirow}
\usepackage{tabularx}
\usepackage{ulem}
\usepackage[utf8]{inputenc}
\usepackage[russian]{babel}
\usepackage{amsmath}
\usepackage{amssymb}

\usepackage{libertine}
\renewcommand{\ttdefault}{cmtt}

\usepackage{titlesec}

\titleformat{\section}
  {\normalfont\Large\bfseries}{\thesection.}{0.3em}{}

\titleformat{\subsection}
  {\normalfont\large\bfseries}{\thesubsection.}{0.3em}{}

\titlespacing{\section}{0pt}{*2}{*2}
\titlespacing{\subsection}{0pt}{*1}{*1}
\titlespacing{\subsubsection}{0pt}{*0}{*0}
\usepackage{listings}
\lstloadlanguages{Lisp}
\lstset{extendedchars=false,
	breaklines=true,
	breakatwhitespace=true,
	keepspaces = true,
	tabsize=2
}

\begin{document}

\section*{Отчет по лабораторной работе №\,4 \\
по курсу \guillemotleft  Функциональное программирование\guillemotright}
\begin{flushright}
Студент группы 8О-307 МАИ \textit{Бирюков Виктор}, \textnumero 2 по списку \\
\makebox[7cm]{Контакты: {\tt vikvladbir@mail.ru} \hfill} \\
\makebox[7cm]{Работа выполнена: 10.05.2022 \hfill} \\
\ \\
Преподаватель: Иванов Дмитрий Анатольевич, доц. каф. 806 \\
\makebox[7cm]{Отчет сдан: \hfill} \\
\makebox[7cm]{Итоговая оценка: \hfill} \\
\makebox[7cm]{Подпись преподавателя: \hfill} \\

\end{flushright}

\section{Тема работы}
Знаки и строки.

\section{Цель работы}
Научиться работать с литерами (знаками) и строками при помощи функций обработки строк и общих функций работы с последовательностями.

\section{Задание (вариант №4.38)}
Запрограммировать на языке Коммон Лисп функцию, принимающую один аргумент --- натуральное число $n$, $n < 1000000$.
Функция должна вернуть предложение, которое выражает это число русскими словами.

\section{Оборудование студента}
Процессор AMD Ryzen 7 3700U\,@\,2.3GHz, память: 20Gb, разрядность системы: 64.

\section{Программное обеспечение}
ОС Windows 10, компилятор SBCL 2.2.2, текстовый редактор Sublime Text 4.

\section{Идея, метод, алгоритм}
Структура количественных числительных в русском языке такова, что цифры числа можно обрабатывать группами по три, при необходимости вставляя после такой группы слова <<тысяча>>, <<миллион>> и т.д.

В пределах группы каждый разряд представляется одним словом, за исключением чисел $11 - 19$, которые обозначаются одним словом на два разряда.

Поправка на падеж нужна в двух случаях. Существует три варианта написания трехразрядных слов --- <<тысяча>> для $1$, <<тысячи>> для $2-4$, <<тысяч>> для $0$, $5-9$, а также $11-19$. Миллионы и более аналогично. Также для тысяч менятся написание: <<один>> --- на <<одна>>, <<два>> --- на <<две>>.

Все необходимые константы записаны в шесть глобальных массивов. Обработку трехзначных групп осуществляет функция {\tt three-digits-sentence}. Обработку всего числа --- функция {\tt number-sentence-iter}, которая описывает линейно-итеративный процесс. Таким образом, сложность алгоритма по времени --- $O(\log_{1000}{n})$

\section{Сценарий выполнения работы}

\section{Распечатка программы и её результаты}

\subsection{Исходный код}
\begin{lstlisting}[escapechar=\%, basicstyle=\footnotesize]
; 4.38

(defvar *digits* #(%"один"% %"два"% %"три"% %"четыре"% %"пять"% %"шесть"% %"семь"% %"восемь"% %"девять"%))
(defvar *digits-for-thousand* #(%"одна"% %"две"%))
(defvar *two-digits* #(%"одиннадцать"% %"двенадцать"% %"тринадцать"% %"четырнадцать"% %"пятнадцать"% %"шестнадцать"% %"семнадцать"% %"восемнадцать"% %"девятнадцать"%))
(defvar *tens* #(%"десять"% %"двадцать"% %"тридцать"% %"сорок"% %"пятьдесят"% %"шестьдесят"% %"семьдесять"% %"восемьдесят"% %"девяносто"%))
(defvar *hundreds* #(%"сто"% %"двести"% %"триста"% %"четыреста"% %"пятьсот"% %"шестьсот"% %"семьсот"% %"восемьсот"% %"девятьсот"%))
(defvar *case* #(#(%"тысяча"% %"тысячи"% %"тысяч"%) #(%"миллион"% %"миллиона"% %"миллионов"%)))

(defun three-digits-sentence (num pos)
    (let ((res NIL))
        (if (= num 0)
            res
            (let 
                ((digit (mod num 10))
                (ten (mod (floor num 10) 10))
                (hundred (mod (floor num 100) 10)))

                (if (/= hundred 0) ; 100 - 900
                    (setq res (concatenate 'string 
                                  res 
                                  (svref *hundreds* (1- hundred)) 
                                  " ")))

                (if (and (= ten 1) (/= digit 0)) ; 11 - 19
                    (setq res (concatenate 'string 
                                  res 
                                  (svref *two-digits* (1- digit)) 
                                  " "))
                    (progn 
                        (if (/= ten 0) ; 10 - 90
                            (setq res (concatenate 'string 
                                          res 
                                          (svref *tens* (1- ten)) 
                                          " ")))
                        (cond
                            ((and (/= digit 0) 
                                  (or (/= pos 1) (> digit 2))) ; 1 - 9
                                (setq res (concatenate 'string 
                                              res 
                                              (svref *digits* (1- digit)) 
                                              " ")))
                            ((/= digit 0) ; 1 - 2 thousand
                                (setq res (concatenate 'string 
                                              res 
                                              (svref *digits-for-thousand* 
                                                     (1- digit)) 
                                              " "))))))
                
                (if (> pos 0)
                    (cond
                        ((or (= digit 0) (= ten 1) (> digit 4)) 
                            (setq res (concatenate 'string 
                                          res 
                                          (svref (svref *case* (1- pos)) 2) 
                                          " ")))
                        ((= digit 1) 
                            (setq res (concatenate 'string 
                                          res 
                                          (svref (svref *case* (1- pos)) 0) 
                                          " ")))
                        (T 
                            (setq res (concatenate 'string 
                                          res 
                                          (svref (svref *case* (1- pos)) 1) 
                                          " ")))))
                res))))

(defun number-sentence-iter (num pos res)
    (if (= num 0)
        res
        (number-sentence-iter
            (floor num 1000)
            (1+ pos)
            (concatenate 'string
                (three-digits-sentence (mod num 1000) pos)
                res))))

(defun number-sentence (num)
    (string-right-trim " " (number-sentence-iter num 0 NIL)))
\end{lstlisting}

\subsection{Результаты работы}
\begin{lstlisting}[escapechar=\%]
* (number-sentence 3967)
%"три тысячи девятьсот шестьдесят семь"%

* (number-sentence 1004012)
%"один миллион четыре тысячи двенадцать"%

* (number-sentence 925101503)
%"девятьсот двадцать пять миллионов сто одна тысяча пятьсот три"%

* (number-sentence 143054637)
%"сто сорок три миллиона пятьдесят четыре тысячи шестьсот тридцать семь"%

* (number-sentence 999999999)
%"девятьсот девяносто девять миллионов девятьсот девяносто девять тысяч девятьсот девяносто девять"%
\end{lstlisting}

\section{Дневник отладки}
\begin{tabular}{|c|c|c|c|}
\hline
Дата & Событие & Действие по исправлению & Примечание \\
\hline
\end{tabular}

\section{Замечания автора по существу работы}
Так как кроме тысяч исключений больше нет, величина обрабатываемых чисел может быть увеличена путем расширения массива {\tt *case*}.

\section{Выводы}
В ходе выполнения лабораторной работы я познакомился со строками и последовательностями в языке Коммон Лисп, а также с функциями для их обработки. 
\end{document}