\documentclass[12pt]{article}

\usepackage{fullpage}
\usepackage{multicol,multirow}
\usepackage{tabularx}
\usepackage{ulem}
\usepackage[utf8]{inputenc}
\usepackage[russian]{babel}
\usepackage{amsmath}
\usepackage{amssymb}

\usepackage{libertine}
\renewcommand{\ttdefault}{cmtt}

\usepackage{titlesec}

\titleformat{\section}
  {\normalfont\Large\bfseries}{\thesection.}{0.3em}{}

\titleformat{\subsection}
  {\normalfont\large\bfseries}{\thesubsection.}{0.3em}{}

\titlespacing{\section}{0pt}{*2}{*2}
\titlespacing{\subsection}{0pt}{*1}{*1}
\titlespacing{\subsubsection}{0pt}{*0}{*0}
\usepackage{listings}
\lstloadlanguages{Lisp}
\lstset{extendedchars=false,
	breaklines=true,
	breakatwhitespace=true,
	keepspaces = true,
	tabsize=2
}

\begin{document}

\section*{Отчет по лабораторной работе №\,2 \\
по курсу \guillemotleft  Функциональное программирование\guillemotright}
\begin{flushright}
Студент группы 8О-307 МАИ \textit{Бирюков Виктор}, \textnumero 2 по списку \\
\makebox[7cm]{Контакты: {\tt vikvladbir@mail.ru} \hfill} \\
\makebox[7cm]{Работа выполнена: 26.03.2022 \hfill} \\
\ \\
Преподаватель: Иванов Дмитрий Анатольевич, доц. каф. 806 \\
\makebox[7cm]{Отчет сдан: \hfill} \\
\makebox[7cm]{Итоговая оценка: \hfill} \\
\makebox[7cm]{Подпись преподавателя: \hfill} \\

\end{flushright}

\section{Тема работы}
Простейшие функции работы со списками Коммон Лисп.

\section{Цель работы}
Научиться конструировать списки, находить элемент в списке, использовать схему линейной и древовидной рекурсии для обхода и реконструкции плоских списков и деревьев.

\section{Задание (вариант №2.36)}
Дан список действительных чисел $(X_1 \ldots X_n)$.
Запрограммируйте рекурсивно на языке Коммон Лисп функцию, которая возвращает
\begin{itemize}
\setlength{\itemsep}{-1mm}
\item сам список, если последовательность $X_1, \ldots, X_n$ упорядочена по убыванию, т.е. $X_1 > X_2 > \ldots > X_n$;
\item список $(X_n \ldots X_1)$ в противном случае.
\end{itemize}

\section{Оборудование студента}
Процессор AMD Ryzen 7 3700U\,@\,2.3GHz, память: 20Gb, разрядность системы: 64.

\section{Программное обеспечение}
ОС Windows 10, компилятор SBCL 2.2.2, текстовый редактор Sublime Text 4.

\section{Идея, метод, алгоритм}
Функция {\tt myreverse} осуществляет обращение списка, аналогично встроенной функции {\tt reverse}. Вспомогательная функция {\tt reverse-iter} описывает линейно-итеративный процесс: пока первый аргумент не пуст, его первый элемент ставится в начало второго аргумента, затем функция вызывается рекурсивно от хвоста первого аргумента и увеличенного второго. Когда первый список опустеет, второй будет содержать его в перевернутом виде.

Функция {\tt decrease-p} проверяет, что элементы списка упорядочены по убыванию. Функция описывает линейно-итеративный процесс: список из менее двух элементов считается упорядоченным, иначе проверяется упорядоченность первых двух элементов, затем рекурсивно --- упорядоченность списка без первого элемента.

Функция {\tt decreasing} проверяет упорядоченность списка функцией {\tt decrease-p} и, если его элементы не убывают, переворачивает список функцией {\tt myreverse}, иначе возвращает исходный.

\section{Сценарий выполнения работы}

\section{Распечатка программы и её результаты}

\subsection{Исходный код}
\lstinputlisting{./lab2.lisp}

\subsection{Результаты работы}
\begin{lstlisting}
* (decreasing (list 9 8 7 6 5 1))
(9 8 7 6 5 1)

* (decreasing (list 9 7 1 5))
(5 1 7 9)

* (decreasing (list 3 3 3 2 1))
(1 2 3 3 3)

* (decreasing (list 1))
(1)

* (decreasing ())
NIL

\end{lstlisting}

\section{Дневник отладки}
\begin{tabular}{|c|c|c|c|}
\hline
Дата & Событие & Действие по исправлению & Примечание \\
\hline
\end{tabular}

\section{Замечания автора по существу работы}
Временная сложность функции --- $O(n)$, в худшем случае --- $O(2 n)$, где $n$ --- длина списка. Если объединить первую и вторую функции, можно уменьшить коэффициент и получить сложность $O(n)$ во всех случаях за счет увеличения пространственной сложности до $O(3 n)$, так как придется дополнительно хранить исходный список.

\section{Выводы}
В ходе выполнения лабораторной работы я познакомился со встроенной структурой данных список. Как и во многих других неимперативных языках, в Коммон Лисп списки имеют структуру, схожую со стековой, что упрощает сложность некоторых операций до константной, сложность же других операций становится линейной.

\end{document}